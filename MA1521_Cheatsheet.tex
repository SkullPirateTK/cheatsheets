\documentclass[10pt,landscape, a4paper]{article}
\usepackage{multicol}
\usepackage{calc}
\usepackage{ifthen}
\usepackage[landscape]{geometry}
\usepackage{amsmath,amsthm,amsfonts,amssymb}
\usepackage{bm}


\pdfinfo{
  /Title (MA1521 Cheatsheet)
  /Creator (Tony Kwong)
  /Producer (pdfTeX 1.40.0)
  /Subject (MA1521 Calculus)
  /Keywords (pdflatex, latex,pdftex,tex)}

% This sets page margins to .5 inch if using letter paper, and to 1cm
% if using A4 paper. (This probably isn't strictly necessary.)
% If using another size paper, use default 1cm margins.
\ifthenelse{\lengthtest { \paperwidth = 11in}}
    { \geometry{top=.5in,left=.5in,right=.5in,bottom=.5in} }
    {\ifthenelse{ \lengthtest{ \paperwidth = 297mm}}
        {\geometry{top=1cm,left=1cm,right=1cm,bottom=1cm} }
        {\geometry{top=1cm,left=1cm,right=1cm,bottom=1cm} }
    }

% Turn off header and footer
\pagestyle{empty}

% Redefine section commands to use less space
\makeatletter
\renewcommand{\section}{\@startsection{section}{1}{0mm}%
                                {-1ex plus -.5ex minus -.2ex}%
                                {0.5ex plus .2ex}%x
                                {\normalfont\large\bfseries}}
\renewcommand{\subsection}{\@startsection{subsection}{2}{0mm}%
                                {-1explus -.5ex minus -.2ex}%
                                {0.5ex plus .2ex}%
                                {\normalfont\normalsize\bfseries}}
\renewcommand{\subsubsection}{\@startsection{subsubsection}{3}{0mm}%
                                {-1ex plus -.5ex minus -.2ex}%
                                {1ex plus .2ex}%
                                {\normalfont\small\bfseries}}
\makeatother

% Define BibTeX command
\def\BibTeX{{\rm B\kern-.05em{\sc i\kern-.025em b}\kern-.08em
    T\kern-.1667em\lower.7ex\hbox{E}\kern-.125emX}}

% print only section numbers
\setcounter{secnumdepth}{1}


\setlength{\parindent}{0pt}
\setlength{\parskip}{0pt plus 0.5ex}

%My Environments
\newtheorem{example}[section]{Example}


% -----------------------------------------------------------------------

\begin{document}
\raggedright
\footnotesize
\begin{multicols}{3}


% multicol parameters
% These lengths are set only within the two main columns
%\setlength{\columnseprule}{0.25pt}
\setlength{\premulticols}{1pt}
\setlength{\postmulticols}{1pt}
\setlength{\multicolsep}{1pt}
\setlength{\columnsep}{2pt}

\begin{flushleft}
\large{
    \underline{MA1521 Cheat Sheet}
    }
\end{flushleft}

% ------------------------------ACTUAL CONTENT-----------------------------------

\section{Pre-calculus}
\subsection{Real Numbers \& Functions}
\begin{gather*}
    a^2 - b^2 = (a + b)(a - b) \\
    |x+y| \leq |x| + |y| \\
    \log_ax = \frac{\ln x}{\ln a} \\
    f:A \longrightarrow B, \hspace{0.5em} 
    g \circ f = g(f(x)), \hspace{0.5em} 
    g \circ f \neq f \circ g
\end{gather*}
A: domain, B: codomain, range: $f = \{f(x) \in B | x \in A\}$ \\
Injective: $f(x) = f(y) \Rightarrow x = y$, surjective: $\forall z \in B, \exists x \in A, f(x) = z$
If $f^{-1}$ exists, then $f$ is bijective

\subsection{Linear Equations}
Slope-intercept: $y = mx + b$ \\
Point-slope: $y - y_1 = m(x + x_1)$ \\
Intercept: $\frac{x}{a} + \frac{y}{b} = 1$ \\
$b$: $y$-intercept, $a$: $x$-intercept, $(x_1, y_1)$ is a point on the line \\
The gradient of the normal of a line is $\frac{1}{m}$

\subsection{Trigonometric Identities}
\begin{gather*}
    \csc x = \frac{1}{\sin x} \\
    \sec x = \frac{1}{\cos x} \\
    \cot x = \frac{1}{\tan x} \\
    \sin ^2 \theta + \cos ^2 \theta = 1 \\
    \tan ^2 \theta + 1 = \sec ^2 \theta \\
    1 + \cot ^2 \theta = \csc ^2 \theta \\
    \sin (A \pm B) = \sin A \cos B \pm \cos A \sin B \\
    \cos (A \pm B) = \cos A \cos B \mp \sin A \sin B \\
    \tan (A \pm B) = \frac{\tan A \pm \tan B}{1 \mp \tan A \tan B} \\
    \sin 2A = 2 \sin A \cos A \\
    \cos 2A = \cos ^ 2 A - \sin ^ 2 A = 2\cos^2A - 1 = 1 - 2\sin^2A \\
    \tan 2A = \frac{2 \tan A}{1- \tan^2 A} \\
    \sin P \pm \sin Q = 2 \sin (\frac{P \pm Q}{2}) \cos (\frac{P \mp Q}{2}) \\
    \cos P + \cos Q = 2 \cos (\frac{P + Q}{2}) \cos (\frac{P - Q}{2}) \\
    \cos P - \cos Q = -2 \sin (\frac{P + Q}{2}) \sin (\frac{P - Q}{2})
\end{gather*}

\subsection{Values of Trigonometric Functions}
\vspace{1ex}
\begin{center}
\bgroup
\def\arraystretch{1.5}
    \begin{tabular}{|c | c | c | c | c | c|}
        \hline
        $\theta$ & 0 & $\frac{\pi}{6}$ & $\frac{\pi}{4}$ & $\frac{\pi}{3}$ & $\frac{\pi}{2}$ \\
        \hline
        $\sin$ & 0 & $\frac{1}{2}$ & $\frac{\sqrt{2}}{2}$ & $\frac{\sqrt{3}}{2}$ & 1 \\
        \hline
        $\cos$ & 1 & $\frac{\sqrt{3}}{2}$ & $\frac{\sqrt{2}}{2}$ & $\frac{1}{2}$ & 0 \\
        \hline
        $\tan$ & 0 & $\frac{\sqrt{3}}{3}$ & 1 & $\sqrt{3}$ & $DNE$ \\
        \hline
    \end{tabular}
\egroup
\end{center}

\subsection{Range of a Function}
First find the maximal domain of $f$ \\
To find the range of $f$, let $y = f(x)$, solve for $x$, deduce the range (exclude values where $f(x)$ is undefined)

\section{Limits}
\subsection{Continuity}
\vspace{1ex}
\begin{center}
    \begin{tabular}{|c | c | c | c | c|}
        \hline
        $c$ & $\lim_{x \to c^-} f(x)$ & $\lim_{x \to c^+} f(x)$ & $\lim_{x \to c} f(x)$ & $f(x)$\\
        \hline
    \end{tabular}
\end{center}
Continuous if $\lim_{x \to c} f(x)$ exists (only if left = right limit) AND \\
Interior point: $\lim_{x \to c} f(x) = f(c)$ \\
Left/right end-point: left/right limit equals $f(c)$ \\
Polynomials, trigonometric/exponential/logarithmic functions and their combinations are continuous

\subsection{Evaluation of Limits}
\vspace{1ex}
\[ \lim_{x \to \pm\infty} \frac{k_{1}x^a}{k_{2}x^b}  = 0 (a < b), \frac{A}{B} (a = b), \pm\infty (a > b) \] \\
\[ \lim_{x \to c} \frac{\sin(g(x))}{g(x)} = \lim_{x \to c} \frac{g(x)}{\sin(g(x))}  = 1\] \\
\[ \lim_{x \to c} \frac{\tan(g(x))}{g(x)} = \lim_{x \to c} \frac{g(x)}{\tan(g(x))}  = 1\] \\
In particular, when $c = 0$ and $g(x) = x$

\subsection{Squeeze Theorem}
If $g(x) \leq f(x) \leq h(x)$ and \[ \lim_{x \to c} g(x) = \lim_{x \to c} h(x) = L \Rightarrow  \lim_{x \to c} f(x) = L \] \\
If $\lim_{x \to c} g(x) = 0$, \[ \lim_{x \to c} g(x)sin(h(x)) = 0, \lim_{x \to c} g(x)cos(h(x)) = 0 \]

\subsection{Intermediate Value Theorem}
To show an equation $f(x) = c$ has a root between $a$ and $b$, $f(x)$ must be continuous and $f(a) < c < f(b)$

\section{Derivatives}
Differentiability implies continuity (converse is not true in general)
\begin{center}
\bgroup
\def\arraystretch{1.5}
    \begin{tabular}{|c | c |}
        \hline
	Function & Derivative \\
        \hline
        $(f(x))^n$ & $nf'(x)(f(x))^{n-1}$ \\
        \hline
        $\cos(f(x))$ & $-f'(x)\sin(f(x))$ \\
        \hline
        $\sin(f(x))$ & $f'(x)\cos(f(x))$ \\
        \hline
        $\tan(f(x))$ & $f'(x)\sec^2(f(x))$ \\
        \hline
        $\sec(f(x))$ & $f'(x)\sec(f(x))\tan(f(x))$ \\
        \hline
        $\csc(f(x))$ & $-f'(x)\csc(f(x))\cot(f(x))$ \\
        \hline
        $\cot(f(x))$ & $-f'(x)\csc^2(f(x))$ \\
        \hline
        $e^{f(x)}$ & $f'(x)e^{f(x)}$ \\
        \hline
        $\ln(f(x))$ & $\frac{f'(x)}{f(x)}$ \\
        \hline
        $\sin^{-1}(f(x))$ & $\frac{f'(x)}{\sqrt{1 - f(x)^2}}$ \\
        \hline
        $\cos^{-1}(f(x))$ & $-\frac{f'(x)}{\sqrt{1 - f(x)^2}}$ \\
        \hline
        $\tan^{-1}(f(x))$ & $\frac{f'(x)}{1 + f(x)^2}$ \\
        \hline
        $\cot^{-1}(f(x))$ & $-\frac{f'(x)}{1 + f(x)^2}$ \\
        \hline
        $\sec^{-1}(f(x))$ & $\frac{f'(x)}{|f(x)|\sqrt{f(x)^2 - 1}}$ \\
        \hline
        $\csc^{-1}(f(x))$ & $-\frac{f'(x)}{|f(x)|\sqrt{f(x)^2 - 1}}$ \\
        \hline
    \end{tabular}
\egroup
\end{center}
\[ \frac{d}{dx}(uv) = \frac{du}{dx}v + u\frac{dv}{dx} \]
\[ \frac{d}{dx}(\frac{u}{v}) = \frac{\frac{du}{dx}v - u\frac{dv}{dx}}{v^2} \]
\[ \frac{d}{dx}(f(g(x))) = f'(g(x)) \cdot g'(x) \]

\subsection{Implicit Differentiation}
Differentiate all terms w.r.t. $x$, chain rule on terms only in $y$ e.g. $\frac{d}{dx}y^3 = 3y^2\frac{dy}{dx}$ \\
For equations of the form $f(x,y) = 0$,
\[ \frac{dy}{dx} = -\frac{\frac{d}{dx}f(x,y)}{\frac{d}{dy}f(x,y)} \]

\subsection{Derivatives of Inverse Functions}
For bijective function $f$,
\[ (f^{-1})'(a) = \frac{1}{f'(f^{-1}(a))} \]
$f'(f^{-1}(a))$ is the derivative of $f$ evaluated at $f^{-1}(a)$

\subsection{Parametric Equations}
For curves defined by the equations $x = f(t), y = g(t)$
\[ \frac{dy}{dx} = \frac{dy}{dt} \div \frac{dx}{dt} \]
\[ \frac{d^{2}y}{dx^2} = \frac{d}{dt}(\frac{dy}{dx}) \div \frac{dx}{dt} \]

\subsection{Concavity, Extremas}
$f''(c) > 0 \Rightarrow$ concave upward / local minima, 
$f''(c) < 0 \Rightarrow$ concave downward / local maxima, 
$f''(c) = 0 \Rightarrow$ point of inflection \\
End-points are not considered to be local extremas \\
Critical point: not an end-point and $f'(c) = 0$ or $DNE$ \\
Absolute extremum: occurs at end-point or critical point

\subsection{L'Hôpital's Rule}
\[ \lim_{x \to c}\frac{f(x)}{g(x)} = \lim_{x \to c}\frac{f'(x)}{g'(x)} \]

\section{Integrals}
\begin{center}
\bgroup
\def\arraystretch{1.5}
    \begin{tabular}{|c | c |}
        \hline
	Function & Integral \\
        \hline
        $\int(ax + b)^n dx$ & $\frac{(ax + b)^{n + 1}}{(n + 1)a} + C, (n \neq 1)$ \\
        \hline
        $\int\frac{1}{ax + b} dx$ & $\frac{1}{a}\ln|ax + b| + C$ \\
        \hline
        $\int e^{ax + b} dx$ & $\frac{1}{a}e^{ax + b} + C$ \\
        \hline
        $\int\sin(ax + b) dx$ & $-\frac{1}{a}\cos(ax + b) + C$ \\
        \hline
        $\int\cos(ax + b) dx$ & $\frac{1}{a}\sin(ax + b) + C$ \\
        \hline
        $\int\tan(ax + b) dx$ & $\frac{1}{a}\ln|\sec(ax + b)| + C$ \\
        \hline
        $\int\sec(ax + b) dx$ & $\frac{1}{a}\ln|\sec(ax + b) + \tan(ax + b)| + C$ \\
        \hline
        $\int\csc(ax + b) dx$ & $-\frac{1}{a}\ln|\csc(ax + b) + \cot(ax + b)| + C$ \\
        \hline
        $\int\cot(ax + b) dx$ & $-\frac{1}{a}\ln|\csc(ax + b)| + C$ \\
        \hline
        $\int\sec^2(ax + b) dx$ & $\frac{1}{a}\tan(ax + b) + C$ \\
        \hline
        $\int\csc^2(ax + b) dx$ & $-\frac{1}{a}\cot(ax + b) + C$ \\
        \hline
        $\int\sec(ax + b)\tan(ax + b) dx$ & $\frac{1}{a}\sec(ax + b) + C$ \\
        \hline
        $\int\csc(ax + b)\cot(ax + b) dx$ & $-\frac{1}{a}\csc(ax + b) + C$ \\
        \hline
        $\int\frac{1}{a^2 + (x + b)^2} dx$ & $\frac{1}{a}\tan^{-1}(\frac{x + b}{a}) + C$ \\
        \hline
        $\int\frac{1}{\sqrt{a^2 - (x + b)^2}} dx$ & $\sin^{-1}(\frac{x + b}{a}) + C$ \\
        \hline
        $\int\frac{-1}{\sqrt{a^2 - (x + b)^2}} dx$ & $\cos^{-1}(\frac{x + b}{a}) + C$ \\
        \hline
        $\int\frac{1}{a^2 - (x + b)^2} dx$ & $\frac{1}{2a}\ln|\frac{x + b + a}{x + b - a}| + C$ \\
        \hline
        $\int\frac{1}{(x + b)^2 - a^2} dx$ & $\frac{1}{2a}\ln|\frac{x + b - a}{x + b + a}| + C$ \\
        \hline
        $\int\frac{1}{\sqrt{(x + b)^2 + a^2}} dx$ & $\ln|(x + b) + \sqrt{(x + b)^2 + a^2}| + C$ \\
        \hline
        $\int\frac{1}{\sqrt{(x + b)^2 - a^2}} dx$ & $\ln|(x + b) + \sqrt{(x + b)^2 - a^2}| + C$ \\
        \hline
        $\int\sqrt{a^2 - x^2} dx$ & $\frac{x}{2}\sqrt{a^2 - x^2} + \frac{a^2}{2}\sin^{-1}\frac{x}{a} + C$ \\
        \hline
        $\int\sqrt{x^2 - a^2} dx$ & $\frac{x}{2}\sqrt{x^2 - a^2} + \frac{a^2}{2}\ln|x + \sqrt{x^2 - a^2}| + C$ \\
        \hline
    \end{tabular}
\egroup
\end{center}

\subsection{\hspace{0.3cm}Useful Identities}
\begin{gather*}
\cos^2A = \frac{1}{2}(1 + \cos2A) \\
\sin^2A = \frac{1}{2}(1 - \cos2A) \\
\sin A \cos B = \frac{1}{2}(\sin(A + B) + \sin(A - B)) \\
\cos A \sin B = \frac{1}{2}(\sin(A + B) - \sin(A - B)) \\
\cos A \cos B = \frac{1}{2}(\cos(A + B) + \cos(A - B)) \\
\sin A \sin B = -\frac{1}{2}(\cos(A + B) - \cos(A - B)) \\
\end{gather*}

\subsection{\hspace{0.3cm}Partial Fractions}
\begin{center}
\bgroup
\def\arraystretch{1.5}
    \begin{tabular}{|c | c |}
        \hline
	Denominator Factors & Partial Fractions \\
        \hline
        $ax + b$ & $\frac{A}{ax + b}$ \\
        \hline
        $(ax + b)^2$ & $\frac{A}{ax + b} + \frac{B}{(ax + b)^2}$ \\
        \hline
        $ax^2 + bx + c, b^2 - 4ac < 0$ & $\frac{Ax + B}{ax^2 + bx + c}$ \\
        \hline
    \end{tabular}
\egroup
\end{center}
\hspace{0.3cm}Improper fraction $\to$ proper fraction: long division

\subsection{\hspace{0.3cm}Integration by Substitution}
\[ \int f(g(x))g'(x) dx = \int f(u) du \]
\begin{center}
\bgroup
\def\arraystretch{1.5}
    \begin{tabular}{|c | c |}
        \hline
	Expression & Trigonometric Substitution \\
        \hline
        $\sqrt{a^2 - (x + b)^2}$ & $x + b = a \sin \theta, -\frac{\pi}{2} \leq \theta \leq \frac{\pi}{2}$ \\
        \hline
        $\sqrt{a^2 + (x + b)^2}$ & $x + b = a \tan \theta, -\frac{\pi}{2} < \theta < \frac{\pi}{2}$ \\
        \hline
        $\sqrt{(x + b)^2 - a^2}$ & $x + b = a \sec \theta, 0 < \theta < \frac{\pi}{2} \hspace{0.25em} or \hspace{0.25em}\pi \leq \theta < \frac{3\pi}{2}$ \\
        \hline
    \end{tabular}
\egroup
\end{center}
\hspace{0.3cm}Manipulate the expression to fit the forms

\subsection{\hspace{0.3cm}Integration by Parts}
\[ \int f'(x)g(x) dx = f(x)g(x) - \int f(x)g'(x) dx \]
\hspace{0.3cm}Choice of integration: \\
\hspace{0.3cm}log, inversion trigo, algebraic functions $\rightarrow$ differentiate \\
\hspace{0.3cm}exponential functions $\rightarrow$ integrate \\
\hspace{0.3cm}trigo functions $\rightarrow$ either

\subsection{\hspace{0.3cm}Fundamental Theorem of Calculus}
\[ \frac{d}{dx} \int^{u(x)}_{a} f(t) dt = f(u(x))u'(x) \]
\hspace{0.3cm}$\int^{a}_{-a} f(x) dx = 0$ if $f(-x) = -f(x)$ (odd function) \\
\hspace{0.3cm}$\int^{a}_{-a} f(x) dx = 2 \int^{a}_{0} f(x) dx$ if $f(-x) = f(x)$ (even function) \\

\subsection{Improper Integrals}
Type 1: integrals with infinite limits of integration \\
\[ \int^{\infty}_{a} f(x) dx = \lim_{b \to \infty} \int^{b}_{a} f(x) dx \]
\[ \int^{b}_{-\infty} f(x) dx = \lim_{a \to -\infty} \int^{b}_{a} f(x) dx \]
\[ \int^{\infty}_{-\infty} f(x) dx = \int^{c}_{-\infty} f(x) dx + \int^{\infty}_{c} f(x) dx \]

Type 2: integrals of functions that become infinite at a point within the interval of integration
\[ \int^{b}_{a} f(x) dx = \lim_{c \to a^+} \int^{b}_{c} f(x) dx \]
$f(x)$ is continuous on $(a, b]$ and is discontinuous at $a$

\subsection{Area Between Curves}
\[ A = \int^{b}_{a} f(x) - g(x) dx \]
$f(x)$ is the curve above $g(x)$, if the two curves alternate between being top and bottom, split the regions into their respective integrals and intervals \\
For curves in terms of $y$, $f(y)$ is further to the right of the $y$-axis than $g(y)$,
\[ A = \int^{d}_{c} f(y) - g(y) dy \]

\subsection{Volume of Solid of Revolution}
Disk method (revolve about the $x$-axis):
\[ V = \pi \int^{b}_{a} f(x)^2 dx - \pi \int^{b}_{a} g(x)^2 dx \]
Use $f(y)$, $g(y)$ and differentiate w.r.t. $y$ for revolution about the $y$-axis \\
Cylindrical shell method (use when difficult/impossible to express $y = f(x)$ as $x = f(y)$):
\[ V = 2\pi \int^{b}_{a} x|f(x) - g(x)| dx \]
The above is for rotation about the \textbf{$\mathbf{y}$-axis}, for rotation about the $x$-axis, use $y|f(y) - g(y)| dy$

\subsection{Arc Length of a Curve}
\[ \int^{b}_{a} \sqrt{1 + f'(x)^2} dx \]

\section{Series}
\subsection{Common Infinite Series}
\subsubsection{Geometric Series}
\[ \sum_{n = 1}^{\infty} ar^{n-1}, \text{\hspace{0.3em}}(a \neq 0) \] \\
\[ \sum_{i = 1}^{n} ar^{i-1} = \frac{a(1 - r^n)}{1 - r} \] \\
Convergent to $\frac{a}{1 - r}$ when $|r| < 1$
\subsubsection{Harmonic Series}
\[ \sum_{n = 1}^{\infty} \frac{1}{n} \text{\hspace{0.3em}is divergent} \]
\subsubsection{Alternating Harmonic Series}
\[ \sum_{n=1}^{\infty} \frac{(-1)^{n+1}}{n} \text{\hspace{0.3em}converges to\hspace{0.3em}} ln2\]
\subsubsection{$\bm{p}$-Series}
\[ \sum_{n = 1}^{\infty} \frac{1}{n^p} \text{\hspace{0.3em}is convergent} \iff p > 1 \]

\subsection{Convergence \& divergence}
\subsubsection{$\bm{n\textsuperscript{th}}$ Term Test}
\[ \text{If\hspace{0.3em}} \sum_{n = 1}^{\infty} a_n \text{\hspace{0.3em}is convergent, then\hspace{0.3em}} \lim_{n \to \infty} a_n = 0 \] \\
Therefore if $\lim_{n \to \infty} a_n$ does not exist or $\lim_{n \to \infty} a_n \neq 0$ then the series is divergent, inconclusive if $\lim_{n \to \infty} a_n = 0$
\subsubsection{Integral Test}
Use when integral is simple/known
\[ \sum_{n = 1}^{\infty} a_n \text{\hspace{0.3em}is convergent} \iff \int^{\infty}_{1} f(x) dx \text{\hspace{0.3em}is convergent} \]
\subsubsection{Comparison Test}
Use when able to establish an inequality to compare to a known series
\[ \sum_{n = 1}^{\infty} a_n, \sum_{n = 1}^{\infty} b_n, \text{\hspace{0.3em}}a_n \leq b_n \text{\hspace{0.3em}for all\hspace{0.3em}} n \] \\
\[ \sum_{n = 1}^{\infty} b_n \text{\hspace{0.3em}is convergent} \rightarrow \sum_{n = 1}^{\infty} a_n \text{\hspace{0.3em}is convergent} \] \\
\[ \sum_{n = 1}^{\infty} a_n \text{\hspace{0.3em}is divergent} \rightarrow \sum_{n = 1}^{\infty} b_n \text{\hspace{0.3em}is divergent} \] 
\subsubsection{Ratio Test and Root Test}
Use root test on series with power $n$ by taking $n^\textsuperscript{th}$ root
\[ \sum_{n = 1}^{\infty} a_n, \lim_{n \to \infty} |\frac{a_{n + 1}}{a_n}| = L \text{\hspace{0.5em}\textbf{OR}\hspace{0.5em}} \lim_{n \to \infty} \sqrt[n]{|a_n|} = L \]
$0 \leq L < 1 \rightarrow$ absolutely convergent \\
$L > 1 \rightarrow$ divergent \\
$L = 1 \rightarrow$ inconclusive

\subsection{Alternating Series}
\[ \sum_{n = 1}^{\infty} (-1)^{n - 1}b_n \] \\
If $b_n$ is decreasing and $\lim_{n \to \infty} b_n = 0$ then the series is convergent

\subsection{Power Series}
Power series centered at $a$ \\
\[ \sum_{n = 0}^{\infty} c_n(x - a)^n = c_0 + c_1(x - a) + c_2(x - a)^2 + ... \] \\
Exactly one of the following: \\
\begin{enumerate}
 \item Converges at $x = a$ only (R = 0)
 \item Converges for all $x$ (R = $\infty$)
 \item There exists a positive number $R$ such that the series converges absolutely if $|x - a| < R$ and diverges if $|x - a| > R$
\end{enumerate}
$R$ is the radius of convergence, end points of the interval of convergence can converge or diverge \\
To compute $R$:
\[ \lim_{n \to \infty} |\frac{c_{n + 1}}{c_n}| = L \text{\hspace{0.5em}\textbf{OR}\hspace{0.5em}} \lim_{n \to \infty} \sqrt[n]{|c_n|} = L, \text{\hspace{0.3em}}L \in \mathbb{R} \vee L = \infty, \text{\hspace{0.3em}}R = \frac{1}{L}\] \\

\subsubsection{Power Series Representation}
\[ f'(x) = \sum_{n=1}^{\infty} n c_{n} (x-a)^{n-1} + C, |x-a| < R \]
\[ \int f(x) dx = \sum_{n=0}^{\infty} c_{n} \frac{(x-a)^{n+1}}{n+1} + C, |x-a| < R \]
To find the power series representation, manipulate the expression into one of the known series to derive the summation

\subsubsection{Taylor Series}
Coefficients of a power series is given as $c_{n} = \frac{f^{(n)}(a)}{n!}$ \\
\[ f(x) = \sum_{n=0}^{\infty} \frac{f^{(n)}(a)}{n!} (x-a)^{n}\]
Power series representation of $f$ at $a$, extension of power series

\subsubsection{Maclaurin Series}
\[ f(x) = \sum_{n=0}^{\infty} \frac{f^{(n)}(0)}{n!} x^{n}\]
Power series representation of $f$ at $0$, special case of Taylor series

\section{Vectors \& Geometry of Space}
Equation of a sphere with center $C(x_1,y_1,z_1)$ and radius $r$ \\
\[ (x-x_1)^2 + (y-y_1)^2 + (z-z_1)^2 = r^2 \]
Distance formula: $|P_1P_2| = \sqrt{(x_2-x_1)^2 + (y_2-y_1)^2 + (z_2-z_1)^2}$ \\
Length of a vector: $||v|| = \sqrt{(v_1)^2 + (v_2)^2 + (v_3)^2}$ \\
Unit vector: $u = v/||v||$ \\
Dot product: $a \cdot b = a_1b_1 + a_2b_2 + a_3b_3$, $a \cdot b = ||a|| ||b|| \cos{\theta}$, $a \cdot a = ||a||^2$ \\
$\theta$ is the angle between $a$ and $b$ \\
 $a$ \& $b$ are perpendicular to each other if $a \cdot b = 0$

\subsection{Projections}
Component of $b$ along $a$: comp$_{a}b = ||b|| \cos{\theta} = \frac{a \cdot b}{||a||}$ \\
Vector projection of $b$ onto $a$: $\text{proj}_{a}b = \text{comp}_{a}b \times \frac{a}{||a||} = \frac{a \cdot b}{||a||^2}a$ \\

\subsubsection{Distance from a point to a plane}
Shortest distance from point $P(x_0, y_0, z_0)$ to the plane $ax + by + cz = d$
\[ \frac{|ax_0 + by_0 + cz_0 - d|}{\sqrt{a^2 + b^2 + c^2}} \]

\subsection{Cross Product}
\[a \times b = (a_2b_3 - a_3b_2)i + (a_3b_1 - a_1b_3)j + (a_1b_2 = a_2b_1)k\] 
$a \times b$ is perpendicular to both $a$ and $b$ \\
Area of a parallelogram: $||a \times b|| = ||a|| ||b|| \sin{\theta}$ \\
Distance from $Q$ to the line through $P$ \& $R$: 
\[||\vec{PQ}|| \sin{\theta} = \frac{||\vec{PQ} \times \vec{PR}||}{||\vec{PR}||}\]

\subsection{Lines}
$r_0$: known point on the line, $t$: scalar multiple, $v$: direction vector of the line \\
Vector equation: $r = r_0 + tv$ or $\langle x, y, z \rangle = \langle x_0, y_0, z_0 \rangle + t \langle a, b, c \rangle $ \\
Parametric equation: $x = x_0 + at$, $y = y_0 + bt$, $z = z_0 + ct$ \\
If two lines in 3D space are not parallel and does not intersect, then they are skew

\subsection{Planes}
$n$: normal vector to the plane, $r_0$: point on the plane
Vector equation: $n \cdot r = n \cdot r_0$ \\
Linear equation: $ax + by + cz + d = 0, d = -(ax_0 + by_0 + cz_0)$ \\
Two planes are parallel if their normal vectors are parallel \\
If two planes are not parallel, they intersect in a line and the angle between two planes is the acute angle between their normal vectors \\

\section{Functions of Several Variables}
\subsection{Vector-valued Function}
$r(t) = \langle f(t), g(t), h(t) \rangle = f(t)i + g(t)j + h(t)k$
$r'(t) = \langle f'(t), g'(t), h'(t) \rangle$

\subsubsection{Derivative Rules}
$f(t)$: differentiable scalar function, $r(t)$ \& $s(t)$: differentiable vector-valued function \\
$\frac{d}{dt} f(t)r(t) = f'(t)r(t) + f(t)r'(t)$ \\
$\frac{d}{dt} r(t) \cdot s(t) = r'(t) \cdot s(t) + r(t) \cdot s'(t)$ \\
$\frac{d}{dt} r(t) \times s(t) = r'(t) \times s(t) + r(t) \times s'(t)$ \\

\subsubsection{Tangent Vector and Tangent Line to a Curve}
$r'(t)$ is the tangent vector of $r(t)$, equation of tangent line can be found using a point on the curve and the corresponding tangent vector

\subsubsection{Arc Length of a Space Curve}
\[ s = \int_{a}^{b} ||r'(t)|| dt \]

\subsection{Functions of Two Variables}
Elliptic paraboloid symmetric about $z$-axis
\[ \frac{x^2}{a^2} + \frac{y^2}{b^2} = \frac{z}{c} \]
Ellipsoid
\[ \frac{x^2}{a^2} + \frac{y^2}{b^2} + \frac{z^2}{c^2} = 1 \]

\subsection{Partial Derivatives}
\subsubsection{Clairaut's Theorem}
$f_{xy}(a,b) = f_{yx}(a,b)$, $f_{xyy}(a,b) = f_{yxy}(a,b) = f_{yyx}(a,b)$

\subsubsection{Equation of Tangent Planes}
$z = f(a,b) + f_x(a,b)(x-a) + f_y(a,b)(y-b)$

\subsubsection{Chain Rule}
One independent variable: $z = f(x,y), x = g(t), y=h(t)$ \\
\[ \frac{dz}{dt} = \frac{\delta f}{\delta x} \frac{dx}{dt} + \frac{\delta f}{\delta y} \frac{dy}{dt} \]
Two independent variables: $z = f(x,y), x = g(s,t), y=h(s,t)$ \\
\[ \frac{dz}{ds} = \frac{\delta f}{\delta x} \frac{\delta x}{\delta s} + \frac{\delta f}{\delta y} \frac{\delta y}{\delta s} \]
\[ \frac{dz}{dt} = \frac{\delta f}{\delta x} \frac{\delta x}{\delta t} + \frac{\delta f}{\delta y} \frac{\delta y}{\delta t} \]

\subsubsection{Implicit Differentiation}
Two independent variables: $F(x,y,z) = 0, z = f(x,y)$ \\
\[ \frac{\delta z}{\delta x} = -\frac{F_x}{F_z}, \frac{\delta z}{\delta y} = -\frac{F_y}{F_z} \]

\subsubsection{Approximation of Increment}
Good approximation provided $\Delta x$ \& $\Delta y$ are small \\
$\Delta z \approx dz = f_x(a,b) \Delta x + f_y(a,b) \Delta y$

\subsection{Directional Derivative}
Gradient of $f(x,y)$
\[ \nabla f(x,y) = \langle f_x, f_y \rangle = f_{x}i + f_{y}j \]
Directional derivative of $f(x,y)$ in the direction of unit vector $u = \langle a, b \rangle$
\[ D_{u}f(x,y) = \nabla f(x,y) \cdot u = f_x(x,y)a + f_y(x,y)b \]
Similarly for $f(x,y,z)$
\[ D_{u}f(x,y,z) = \nabla f(x,y,z) \cdot u = f_x(x,y,z)a + f_y(x,y,z)b + f_z(x,y,z)c \]

\subsubsection{Tangent Plane to Level Surface}
$\nabla f$ is normal to the level curve/surface $f$ \\
\[ \nabla f(x_0,y_0,z_0) \cdot \langle x-x_0, y-y_0, z-z_0 \rangle = 0 \]

\subsubsection{Maximum Rate of Increase/Decrease}
Maximum value of $D_{u}f(P)$:  $||\nabla f(P)||$, minimum value: $-||\nabla f(P)||$

\subsection{Local Extrema}
If $f(x,y)$ has a local maximum or minimum at $(a,b)$, then $f_x(a,b) = f_y(a,b) = 0$ \\
$(a,b)$ is a critical point if the above is true or one of the partial derivatives does not exist \\

\subsubsection{Second Derivative Test}
\[ D(a,b) = f_{xx}(a,b)f_{yy}(a,b) - (f_{xy}(a,b))^2 \]
$f_{yy}$ can be substituted as $f_{xx}$ below \\
\begin{itemize}
\item Local maximum: $D > 0$ and $f_{xx}(a,b) > 0$ \\
\item Local minimum: $D > 0$ and $f_{xx}(a,b) < 0$ \\
\item Saddle point: $D < 0$ \\
\item No conclusion: $D = 0$ \\
\end{itemize}

\section{Double Integrals}
\subsection{Iterated Integral}
\[ V = \int\int_{R} f(x,y) dA = \int_{a}^{b} \int_{c}^{d} f(x,y) dydx = \int_{c}^{d} \int_{a}^{b} f(x,y) dxdy \]
If $f(x,y)$ can be factored into $g(x)h(y)$
\[ \int\int_{R} g(x)h(y) dA = (\int_{a}^{b} g(x) dx)(\int_{c}^{d} h(y) dy)\]

\subsubsection{Double Integral Over a General Region}
Type I domain, region between two functions of $x$
\[ \int_{a}^{b} \int_{g_1(x)}^{g_2(x)} f(x,y) dydx, y = g_1(x), y = g_2(x) \]
Type II domain, region between two functions of $y$
\[ \int_{c}^{d} \int_{h_1(y)}^{h_2(y)} f(x,y) dxdy, x = h_1(y), x = h_2(y) \]

\subsection{Double Integrals in Polar Coordinates}
$r^2 = x^2 + y^2, x = r\cos{\theta}, y = r\cos{\theta}$
\[ \int\int_{R} f(x,y) dA = \int_{\alpha}^{\beta} \int_{a}^{b} f(r\cos{\theta}, r\sin{\theta}) rdrd\theta \]
$0 \leq a \leq r \leq b, \alpha \leq \theta \leq \beta$, sketch $R$ to determine limits, use polar coordinates when region is circular

\subsection{Surface Area}
\[ \int\int_{D} \sqrt{(f_x)^2 + (f_y)^2 + 1} \text{\hspace{0.3em}} dA \]

\section{Ordinary Differential Equations}
\subsection{Separable ODE}
\[ \frac{dy}{dx} = f(x)g(y) \]
Separate the variables \\
\[ \frac{1}{g(y)}dy = f(x)dx \]
Integrate both sides \\
\[ \int \frac{1}{g(y)}dy = \int f(x)dx + C\]

\subsection{Reduction to Separable Form}
\subsubsection{Radical}
\[ y' = g(\frac{y}{x}) \]
Let $v = \frac{y}{x}$, then $y=vx$ and $y' = v + xv'$, and the equation becomes $v + xv' = g(v)$ which is separable \\

\subsubsection{Linear}
\[ y' = f(ax + by + c) \]
Set $u = ax + by + c$ \\

\subsection{Linear First Order ODE}
\[ \frac{dy}{dx} + P(x)y = Q(x) \]
Integrating factor: $I(x) = e^{\int P(x) dx}$ \\
Multiply both sides by $I(x)$ and integrate both sides, solve for y \\
\[ y \cdot I(x) = \int Q(x) \cdot I(x) dx \]

\subsection{Bernoulli Equation}
\[ y' + p(x)y = q(x)y^n \]
Let $u = y^{1-n}$, then $u' = (1-n)y^{1-n}y'$, so
\[ u' + (1-n)p(x)u = (1-n)q(x) \]
Which is a linear first order ODE

\subsection{Improved Malthus Model of Population}
\[ N = \frac{N_{\infty}}{1 + (\frac{N_{\infty}}{N} - 1)e^{-Bt}} \]
$D = sN$, $N_{\infty} = \frac{B}{s}$ (carrying capacity), $N = \frac{B}{2s}$ (point of inflection) \\

\subsection{Common ODE}
\subsubsection{Half-life of $x$}
Let $x(t)$ be the amount of $x$ at time t, $\frac{dx}{dt} = -k_{x}x(t)$ $\rightarrow$ $x(t) = x_0 e^{-k_{x}t}$ $\rightarrow$ $k_x = \frac{ln2}{\text{half-life of x}}$

\subsubsection{Newton's Law of Cooling}
\[ \frac{dT}{dt} = -k(T - T_{\text{amb}}) \]

\subsubsection{Newton's Second Law}
\[ \text{force} - \text{resistance} = m \frac{dV}{dt} \]

\subsubsection{Mixture}
\[ \frac{dQ}{dt} = \text{input} - \text{output} \]

\end{multicols}
\end{document}
