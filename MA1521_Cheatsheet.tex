\documentclass[10pt,landscape, a4paper]{article}
\usepackage{multicol}
\usepackage{calc}
\usepackage{ifthen}
\usepackage[landscape]{geometry}
\usepackage{amsmath,amsthm,amsfonts,amssymb}
\usepackage{color,graphicx,overpic}
\usepackage{hyperref}
\usepackage{listings}
\usepackage{color}
\usepackage{physics}


\pdfinfo{
  /Title (MA1521 Cheatsheet)
  /Creator (TeX)
  /Producer (pdfTeX 1.40.0)
  /Subject (Example)
  /Keywords (pdflatex, latex,pdftex,tex)}

% This sets page margins to .5 inch if using letter paper, and to 1cm
% if using A4 paper. (This probably isn't strictly necessary.)
% If using another size paper, use default 1cm margins.
\ifthenelse{\lengthtest { \paperwidth = 11in}}
    { \geometry{top=.5in,left=.5in,right=.5in,bottom=.5in} }
    {\ifthenelse{ \lengthtest{ \paperwidth = 297mm}}
        {\geometry{top=1cm,left=1cm,right=1cm,bottom=1cm} }
        {\geometry{top=1cm,left=1cm,right=1cm,bottom=1cm} }
    }

% Turn off header and footer
\pagestyle{empty}

% Redefine section commands to use less space
\makeatletter
\renewcommand{\section}{\@startsection{section}{1}{0mm}%
                                {-1ex plus -.5ex minus -.2ex}%
                                {0.5ex plus .2ex}%x
                                {\normalfont\large\bfseries}}
\renewcommand{\subsection}{\@startsection{subsection}{2}{0mm}%
                                {-1explus -.5ex minus -.2ex}%
                                {0.5ex plus .2ex}%
                                {\normalfont\normalsize\bfseries}}
\renewcommand{\subsubsection}{\@startsection{subsubsection}{3}{0mm}%
                                {-1ex plus -.5ex minus -.2ex}%
                                {1ex plus .2ex}%
                                {\normalfont\small\bfseries}}
\makeatother

% Define BibTeX command
\def\BibTeX{{\rm B\kern-.05em{\sc i\kern-.025em b}\kern-.08em
    T\kern-.1667em\lower.7ex\hbox{E}\kern-.125emX}}

% print only section numbers
\setcounter{secnumdepth}{1}


\setlength{\parindent}{0pt}
\setlength{\parskip}{0pt plus 0.5ex}

%My Environments
\newtheorem{example}[section]{Example}


% -----------------------------------------------------------------------

\begin{document}
\raggedright
\footnotesize
\begin{multicols}{3}


% multicol parameters
% These lengths are set only within the two main columns
%\setlength{\columnseprule}{0.25pt}
\setlength{\premulticols}{1pt}
\setlength{\postmulticols}{1pt}
\setlength{\multicolsep}{1pt}
\setlength{\columnsep}{2pt}

\begin{flushleft}
\large{
    \underline{MA1521 Cheat Sheet}
    }
\end{flushleft}

% ------------------------------ACTUAL CONTENT-----------------------------------

\section{Pre-calculus}
\subsection{Real Numbers \& Functions}
\begin{gather*}
    a^2 - b^2 = (a + b)(a - b) \\
    |x+y| \leq |x| + |y| \\
    f:A \longrightarrow B, \hspace{0.5em} 
    g \circ f = g(f(x)), \hspace{0.5em} 
    g \circ f \neq f \circ g
\end{gather*}
A: domain, B: codomain, range: $f = \{f(x) \in B | x \in A\}$ \\
Injective: $f(x) = f(y) \Rightarrow x = y$, surjective: $\forall z \in B, \exists x \in A, f(x) = z$
If $f^{-1}$ exists, then $f$ is bijective

\subsection{Trigonometric Identities}
\begin{gather*}
    \csc x = \frac{1}{\sin x} \\
    \sec x = \frac{1}{\cos x} \\
    \cot x = \frac{1}{\tan x} \\
    \sin ^2 \theta + \cos ^2 \theta = 1 \\
    \tan ^2 \theta + 1 = \sec ^2 \theta \\
    1 + \cot ^2 \theta = \csc ^2 \theta \\
    \sin (A \pm B) = \sin A \cos B \pm \cos A \sin B \\
    \cos (A \pm B) = \cos A \cos B \mp \sin A \sin B \\
    \tan (A \pm B) = \frac{\tan A \pm \tan B}{1 \mp \tan A \tan B} \\
    \sin 2A = 2 \sin A \cos A \\
    \cos 2A = \cos ^ 2 A - \sin ^ 2 A = 2\cos^2A - 1 = 1 - 2\sin^2A \\
    \tan 2A = \frac{2 \tan A}{1- \tan^2 A} \\
    \sin P \pm \sin Q = 2 \sin (\frac{P \pm Q}{2}) \cos (\frac{P \mp Q}{2}) \\
    \cos P + \cos Q = 2 \cos (\frac{P + Q}{2}) \cos (\frac{P - Q}{2}) \\
    \cos P - \cos Q = -2 \sin (\frac{P + Q}{2}) \sin (\frac{P - Q}{2})
\end{gather*}

\subsection{Values of Trigonometric Functions}
\vspace{1ex}
\begin{center}
    \begin{tabular}{|c c c c c c|}
        \hline
        $\theta$ & 0 & $\frac{\pi}{6}$ & $\frac{\pi}{4}$ & $\frac{\pi}{3}$ & $\frac{\pi}{2}$ \\
        \hline
        $\sin$ & 0 & $\frac{1}{2}$ & $\frac{\sqrt{2}}{2}$ & $\frac{\sqrt{3}}{2}$ & 1 \\
        \hline
        $\cos$ & 1 & $\frac{\sqrt{3}}{2}$ & $\frac{\sqrt{2}}{2}$ & $\frac{1}{2}$ & 0 \\
        \hline
        $\tan$ & 0 & $\frac{\sqrt{3}}{3}$ & 1 & $\sqrt{3}$ & $DNE$ \\
        \hline
    \end{tabular}
\end{center}

\subsection{Range of a Function}
First find the maximal domain of $f$ \\
To find the range of $f$, let $y = f(x)$, solve for $x$, deduce the range (exclude values where $f(x)$ is undefined)

\end{multicols}
\end{document}
